\documentclass{article}
\usepackage{amsmath}
\usepackage{gensymb}
\usepackage{floatflt}
\setcounter{secnumdepth}{5}
\setlength{\textwidth}{6.5in}
\usepackage{setspace}
\usepackage{geometry}
\geometry{left=10mm,right=10mm}
\begin{document}
These are some of the doubts that i've had while deriving the equations.
While we derived the stress balance equation at the liquid -gas interface, we did the following,
\begin{itemize}
\item[1.]

\end{itemize}

\begin{equation}
\bigg[(-p + 2\mu \frac{\partial u_{r}}{\partial r}) + 
 \mu \bigg(\frac{1}{r}\frac{\partial u_{r}}{\partial \theta} -\frac{u_{\theta }}{r}+\frac{\partial u_{\theta }}{\partial r}\bigg)(\frac{-1}{r}\frac{\partial h}{\partial \theta}) + \mu \bigg(\frac{\partial u_{z}}{\partial r} + \frac{\partial u_{r}}{\partial z}\bigg)(\frac{-\partial h}{\partial z})\bigg]
 
\end{equation}

\begin{equation*}
\frac{1}{\bigg(\sqrt{\bigg(\frac{1}{r}\frac{\partial h}{\partial \theta}\bigg)^2 +  1}\bigg)}\frac{1}{\sqrt{\bigg(\frac{\partial h}{\partial z}\bigg)^2 + \bigg(\frac{1}{r}\frac{\partial h}{\partial \theta}\bigg)^2 + 1}}
\end{equation*}



\end{document}